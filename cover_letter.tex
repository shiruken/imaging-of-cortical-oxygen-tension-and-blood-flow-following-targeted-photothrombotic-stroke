\documentclass[12pt,stdletter,orderfromtodate,dateleft,sigleft]{newlfm}
\usepackage{xfrac}

\topmarginsize{0in}
\topmarginskip{0in}
\headermarginsize{0in}
\headermarginskip{0in}

\leftmarginsize{1in}
\rightmarginsize{1in}

\makeatletter
    \newcommand{\pO}{\ensuremath{\text{pO}_2}}
\makeatother

\newlfmP{addrfromskipbefore=0pt}
\newlfmP{addrfromskipafter=0pt}

\newlfmP{sigskipbefore=10pt}
\newlfmP{sigsize=40pt}

\newlfmP{Headlinewd=0pt,Footlinewd=0pt}

\namefrom{Colin Sullender}
\addrfrom{%
    Colin Sullender\\
    The University of Texas at Austin\\
    Department of Biomedical Engineering\\
    107 W. Dean Keeton Street, Stop C0800\\
    Austin, TX, 78712, USA
}

\dateset{\today}

\greetto{Dear Dr. Boas,}

\closeline{Sincerely,}

\begin{document}
\begin{newlfm}

We wish to submit a new manuscript entitled ''Chronic imaging of cortical oxygen tension and blood flow after targeted vascular occlusion'' for consideration by \emph{Neurophotonics}. We confirm that this work is original and has not been published elsewhere nor is it currently under consideration in another journal.

In this paper, we report on the development of an optical imaging system enabling the chronic study of cortical oxygen tension (\pO) and blood flow following ischemic stroke in mice. The system combines phosphorescence lifetime quenching with laser speckle contrast imaging to provide acute and chronic measurements of cortical hemodynamics. A digital micromirror device (DMD) is used to localize phosphorescence excitation and to induce targeted photothrombotic occlusions within vasculature. We use the system to examine the effects of excitation wavelength and penetration depth on measured phosphorescence lifetime. We dynamically monitor \pO\ and blood flow during targeted photothrombosis and highlight the propagation of an ischemia-induced depolarization event. We track the progression of an ischemic lesion over the course of eight days and witness a rapid recovery in both \pO\ and blood flow.

This work is significant because it describes a versatile, dual-modality imaging system capable of inducing and chronically tracking ischemic stroke. We believe this is the first demonstration of targeted photothrombosis being used to create an extended lesion within a singular vessel and the first reporting of vascular \pO\ in the cortex during an ischemic depolarization event. This work should be of interest to readers in the areas of optical imaging, neuroscience, stroke, and animal models of ischemic stroke.

Please address all correspondence to Dr. Andrew Dunn at \underline{adunn@utexas.edu}.

Thank you for your consideration.

\end{newlfm}
\end{document}
